\documentclass{article}

\usepackage[utf8]{inputenc}

\usepackage{amsmath}

\title{Control Theory Home Work 1}
\author{Utkarsh Kalra BS18-03 }
\date{Varient g}

\usepackage{natbib}
\usepackage{graphicx}

\begin{document}

\maketitle

\section{Git repo}
https://github.com/kalraUtkarsh/Control-Theory-Utkarsh-Kalra


\section{TRANSFER FUNCTION CALCULATIONS}
\includegraphics[scale =0.7]{Capture}
$
W1 =\frac{2}{s+5}\\*
$
$
W2 = \frac{s+1}{s+0.5}\\*
$
$
W3 = \frac{1}{s+0.25}\\*
$
$
W4 = \frac{1}{2s+3}\\*
$
\subsection{2(A)}
\includegraphics[scale=0.5]{2}
$(xW1 - yW4)W2 = y\\*\\*
xW1W2 - yW4W2\\*\\*
xW1W2 = y(1+W4W2)\\*\\*
$$
y = \frac{xW1W2}{1+W4W2}\\*\\*
$$
W = {\displaystyle\frac{W1*W2*W3}{1+W4W2}}\\*
$
$
W1*W2*W3 = {\displaystyle\frac{2(s+1)}{(s+5)(s+0.5)(s+0.25)}}\\*\\*\\*
$
$
1+W4W2 = {\displaystyle\frac{s+1+(s+0.5)(2s+3)}{(s+0.5)(2s+3)}}\\*\\*\\*
$
$
final W = {\displaystyle\frac{32s^2+80s+48}{16s^4+124s^3+250s^2+155s+25}}\\*\\*\\*$
\subsection{2(B)}
\subsubsection{Plot with the step}
(There are 3 lines in the graph: blue for the input and yellow and red for the 2 systems of transfer functions but they overlap)\\*
\includegraphics[scale = 0.8]{20}
\includegraphics[scale=0.7]{20_2}
\subsubsection{Plot with the frequency}
(There are 3 lines in the graph: blue for the input and yellow and red for the 2 systems of transfer functions but they overlap)\\*
\includegraphics[scale = 0.8]{21}
\includegraphics[scale=0.7]{21_2}
\subsubsection{Plot with Impulse}
(There are 3 lines in the graph: blue for the input and yellow and red for the 2 systems of transfer functions but they overlap)\\*
\includegraphics[scale=0.8]{22}
\includegraphics[scale=0.7]{22_2.PNG}
\subsection{2(C)}
\subsubsection{Taking the step input and generating the bode plot and pole zero plot.}
The bode plot:\\*
\includegraphics[scale=0.7]{bodebete}\\*
The pole zero Plot: \\*
\includegraphics[scale=0.7]{pole_zerobete}\\*
The given System is Stable as the bode plot is converging and the Phase margins are positive. 
\subsection{2(D): Analyzing the Bode plot, calculating the asymptotes and frequency breaks}
our transfer function:${\displaystyle\frac{xW1W2}{1+W4W2}}$\\*
 = $\displaystyle\frac{\frac{2}{s+5}\frac{s+1}{s+0.5}\frac{1}{s+0.25}}{1+\frac{s+1}{s+0.5}\frac{1}{2s+3}}$
 = ${\displaystyle\frac{48 (s+1)(\frac{s}{1.5}+1)}{25(\frac{s}{5}+1)(\frac{s}{0.25}+1)(\frac{s}{\frac{5-\sqrt{5}}{4}+1})(\frac{s}{\frac{5+\sqrt{5}}{4}+1})}} $
 
 This shows that we have 2 zeroes at frequencies 1.5 and 1 and 4 poles at frequencies 5,0.25,$\frac{5-\sqrt{5}}{4}$ and $\frac{5-\sqrt{5}}{4}$ Therefore there are 6 break frequencies.
 Asymptote 1 is a horizontal line through magnitude $A_{0} = 20log(48/25) \approx 5.57 db$
 As we know that we can state that the next asymptotes slopes increases or decreases 20db/dec in comparison to the previous ones. And using the formula Anext = slope * log($\frac{Wnext}{Wprev}$) + Aprev, we have the table of asymptotes for magnitude plot as below.\\*
 \begin{tabular}{ |p{1cm}||p{3cm}|p{3cm}|p{3cm}|  }
 \hline
 \multicolumn{4}{|r|}{Corner(point passed through)} \\
 \hline
 Sr.& Slope(db/dec) &Frequency(rad/sec)& Magnitude(db)\\
 \hline
 1   & 0    &0.25&   5.57\\
 1 &   -20  & 0.25   &5.57\\
 2 & -40 & 0.69&  -3.25\\
 3    &-20 & 1&  -9.7\\
 4&   0  & 1.5 & -13.22\\
 5& -20  & 1.81   &-13.22\\
 6& -40  & 5 &-22.05\\
 \hline
\end{tabular}\\*
As from the bode plot previously, we got gain crossover frequency of the system as 3.777rad/sec. So, the magnitude plot and frequency axis intersect at (3.777,0).

\section{Total transfer function of the given loop}
$W(s) = {\displaystyle\frac{2}{s^2+2}}$\\*
$M(s) = {\displaystyle\frac{s+2}{2s+3}}$\\*
\includegraphics[]{3}\\*
By the formula given in the lab:\\*
\includegraphics[scale = 0.4]{4}\\*\\*
$x = {\displaystyle\frac{g(t)W(s)+f(t)M(s)}{1+W(s)}}$\\*\\*\\*\\*
$x = {\displaystyle\frac{g(t)\frac{2}{s^2+2}+f(t)\frac{s+2}{2s+3}}{1+\frac{2}{s^2+2}}}$\\*
\section{Finding the Transfer Function from the State Space Representation }
\includegraphics[]{15}\\*
As given in the lab for converting from the State Space to Transfer function the following is used:\\*
\includegraphics[scale = 0.5]{16}
so for our given Matrices:\\*

$
\begin{bmatrix}
1 & 3\\
\end{bmatrix}
$
(
$
\begin{bmatrix}
s & 0\\
0 & s\\
\end{bmatrix}
$
-
$
\begin{bmatrix}
3 & 1\\
-2 & 2\\
\end{bmatrix}
$
$)^-1$
*
$
\begin{bmatrix}
2\\
0\\
\end{bmatrix}
$
+
$
\begin{bmatrix}
1\\
\end{bmatrix}
$
By using the ss2tf function in matlab we get:\\*
\includegraphics[scale =0.8]{18_4}\\*
Therefore the Transfer funtion is:\\*
$
TF = {\displaystyle\frac{s^2-3s-8}{s^2-5s+8}}
$
\section{Finding the Transfer function from the given State Space Representation}
\includegraphics[]{17}\\*\\*\\*\\*\\*\\*\\*\\*
As the given D Matrix has two columns this means that there will be Two tranfer functions as there are Two inputs\\*\\*\\
And to convert SS to TF the formula is:\\*\\*\\*
\includegraphics[scale= 0.5]{16}\\*
So for our given matrices:
$
\begin{bmatrix}
1 & 1\\
\end{bmatrix}
$
(
$
\begin{bmatrix}
s & 0\\
0 & s\\
\end{bmatrix}
$
-
$
\begin{bmatrix}
5 & 1\\
0 & -2\\
\end{bmatrix}
$
$)^-1$
*
$
\begin{bmatrix}
0 & 2\\
2 & 3\\
\end{bmatrix}
$
+
$
\begin{bmatrix}
1 & 6\\
\end{bmatrix}
$
By using the ss2tf function in matlab we get:\\*
For first Input:\\*
\includegraphics[scale=0.8]{19_1}\\*
So the TF for this input is:\\*
$
TF = {\displaystyle\frac{s^2-s-18}{s^2-3s-10}}\\*
$
For Second input:\\*
\includegraphics[scale =0.8]{19_2}\\*
So the TF for this input is:
$
TF = {\displaystyle\frac{6s^2-13s-68}{s^2-3s-10}}\\*
$
\section{Simplifying the system step by step for both the Inputs x and f}
\includegraphics[scale = 0.7]{5}\\*
Step 1\\*
\includegraphics[scale = 0.7]{ss1_1.PNG} \\*
Step 2\\*
\includegraphics[scale = 0.7]{ss2.PNG} \\*
Step 3\\*
\includegraphics[scale = 0.7]{ss3.PNG}\\*\\*\\*\\*


Step 4\\*
\includegraphics[scale = 0.7]{ss4.PNG}\\*
Step 5\\*
\includegraphics[scale = 0.7]{ss5.PNG}\\*

Final answers:\\*
\includegraphics[scale = 0.7]{ss6.PNG}\\*

\end{document}
