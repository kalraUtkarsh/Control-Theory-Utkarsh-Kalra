\documentclass{article}
\usepackage{graphicx}
\graphicspath{ {./images/} }
\usepackage[utf8]{inputenc}
\usepackage[margin=0.5in]{geometry}
\usepackage{amsmath}

\usepackage{hyperref}
\title{Control Theory Home Work 5}
\author{Utkarsh Kalra BS18-03 \and u.kalra@innopolis.university }

\date{Varient g}

\usepackage{natbib}
\usepackage{graphicx}

\begin{document}

\maketitle

\section{Git repo}
https://github.com/kalraUtkarsh/Control-Theory-Utkarsh-Kalra

\section{Observers}
\large
We have already already defined the dynamics and other things in the previous homework as the system is the same as in the previous one.\\*
The dynamics:\\*
$
\centering\large(M+m)x''-mlcos(\theta)\theta''+mlsin(\theta)\theta^2 \\*
\theta''-cos(\theta)x''-gsin(\theta)=0\\*
$\\*
The system in the state space form:\\*
$
z' = f(z) + g(z)u\\*
y = h(z) = 
\begin{bmatrix}
x & (\theta)^2\\
\end{bmatrix}^T\\*
where z = 
\begin{bmatrix}
x & (\theta) & x' & (\theta)'\\
\end{bmatrix}^T
$
\\*
And as done on the previous homework in task 2(B)\\*
$
\displaystyle
\begin{bmatrix}
x'\\
\theta'\\
x''\\
\theta''\\
\end{bmatrix}
=
\begin{bmatrix}
x'\\
\theta'\\
\frac{-mlsin(\theta)\theta'^2+mgsin(\theta)cos(\theta)}{(M+m)-mcos^2(\theta)}\\
\frac{-(M+m)gsin(\theta)+mlcos(\theta)sin(\theta)\theta'^2}{mlcos^2(\theta)-(M+m)l}\\
\end{bmatrix}
+
\begin{bmatrix}
0\\
0\\
\frac{1}{(M+m)-mcos^2(\theta)}\\
\frac{cos(\theta)}{(M+m)-mcos^2(\theta)}\\
\end{bmatrix}
$
\\*
And as done in the 2(C) of the last homework
A = \\*
\includegraphics[]{5_1.PNG}\\*
B = \includegraphics[]{5_2}\\*
My varient \textbf{g}, M=11.6, m=2.7, l=0.57\\*
$
A=
\begin{bmatrix}
0&0&1&0\\
0&0&0&1\\
0&2.2810&-0.0862&0\\
0&-207.7090&-0.15124&0\\
\end{bmatrix}
B=
\begin{bmatrix}
0\\
0\\
0.0862\\
0.15124\\
\end{bmatrix}\\*
C = 
\begin{bmatrix}
1&0&0&0\\
0&1&0&0\\
\end{bmatrix}
$
\\*
\subsection{2(A)}
for this we have to check the rank of the observability matrix, and check the number of unobservable states. \\*
the matrix has full rank and the unobservable states comes to be 0 in matlab, so the system is \textbf{OBSERVABLE}.
\subsection{2(B)}
For this we have to calculate the eigenvalues of A, and if all the values are negative then the system is stable and eigenvalues of A in my case comes out to be: \includegraphics[]{5_3.PNG}\\*
not all values are negative so the system is \textbf{not stable}
\subsection{2(C)LUENBERGER OBSERVER}
Pole placement: there will be the estimation state $\hat{z}$, error by e = z -$\hat{z} $\\*
{\centering\includegraphics[]{5_4.PNG}}\\*
the error is given by the poles A-LC, 
As A is 4x4 matrix we have to have 4 poles, get the L matrix and check the result.\\*
\includegraphics[scale=0.62]{5_5_1.PNG}
\includegraphics[scale=0.62]{5_6_1.PNG}
\\*
The first figure shows the unstable system system and the second figure shows the stable after adding the observer still its not staedy yet.\\*
\newpage
now doing the LQR\\*
for this we need the Q and R matrices and they both will be 1
Q = 
$
\begin{bmatrix}
1&0&0&0\\
0&1&0&0\\
0&0&1&0\\
0&0&0&1\\
\end{bmatrix}
$\\*
R=1\\*
\includegraphics[]{5_7.PNG}\\*
this figure shows that the system does become stable after some time but still fluctuates.\\*

\subsection{2(D)State Feedback Controller}
it is denoted by: u= r$k_{r} - k_{z}$\\*
so the system is\\*
$
\.{z} = (A-BK)z + Brk_{r} = A_[d]z + Brk_{R}\\*
y = Cz\\*
u = rk_{r} - Kz
$
\\*
using pole placement method for designing the state feedback controller\\*
\includegraphics[]{5_9.PNG}\\*

\includegraphics[]{abhi_state.PNG}\\*
Here both the scaled and unscaled systems follow kind of a similar pattern, yet for scaled the gain kr helps reach the steady state of the system.
\subsection{2(E)Luenberger and State Feedback Controller}
Here we are supposed to have both observer and controller in the system:\\*
$
\hat{z} = A\hat{z} + Bu + L(y-\hat{y})\\*
\hat{y} = C\hat{z}\\*
u = rk_{r}-K\hat{z}\\*
So\\*
\.{\hat{z}} = A\hat{z}+B(rk_[r]-K\hat{z})+L(z-\hat{z})\\*
\.{e} = (A - BK - LC)e\\*
$\\*
\includegraphics[]{5_(e).PNG}\\*

\includegraphics[]{5_10(e).PNG}\\*
\subsection{2(F)Gaussian Noise}
Gaussian noise is s statistical noise having a probability density function equal to that of the normal distribution, which is also known as the Gaussian distribution.\\*
And now we need to add that to the output, so the system becomes:\\*
$
\.{z} = Az + bu\\*
y = Cz + v\\*
$
We have to consider the noise as a new input so naturally B and D matrices have to be changed.\\*
$
u_{a} = \begin{bmatrix}
u&v\\
\end{bmatrix}^T\\*
\.{z} = Az + 
\begin{bmatrix}
B&0\\
\end{bmatrix}
\begin{bmatrix}
u&v\\
\end{bmatrix}^T = Az + Bu\\*
y=Cz+\begin{bmatrix}
D&1\\
\end{bmatrix}
\begin{bmatrix}
u&v\\
\end{bmatrix}^T = Cz + v
$\\*
Now let the noise v be some random vector.\\*
\includegraphics[]{fwala.PNG}\\*

\includegraphics[]{thisisf.PNG}\\*
Now the graph is super noisy
\subsection{2(G)}
$
\.{z} = Az + bu + w\\*
y = Cz + v\\*
$
similar steps as in the previous part.\\*
$
u_{a} = \begin{bmatrix}
u&v&w\\
\end{bmatrix}^T\\*
\.{z} = Az + 
\begin{bmatrix}
B&0&1\\
\end{bmatrix}
\begin{bmatrix}
u&v&w\\
\end{bmatrix}^T = Az + Bu + w\\*
y=Cz+\begin{bmatrix}
D&1&0\\
\end{bmatrix}
\begin{bmatrix}
u&v&w\\
\end{bmatrix}^T = Cz + v
$\\*
\includegraphics[]{thiit.PNG}\\*

\subsection{Kalman filter}
The whole explaination of how to implement the kalman filter with matlab and what does it do is given \href{https://elib.uni-stuttgart.de/bitstream/11682/3709/1/kleinbauer.pdf}{here}.\\*
\includegraphics[]{5_h_code.PNG}\\*

\includegraphics[scale=0.8]{sahi kalman.PNG}\\*
In this kalman is blue, the real system is red and green is the noisy system therefore kalman is pretty close to the original system.
\subsection{2(J)LGQ}
\includegraphics[]{5_(j).PNG}
\end{document}
